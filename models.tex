%\hal{i'm having a hard time understanding what's new and what's prior work here.}

\kathy{This section feels very long. Do you think all the formulas are needed? They take up a lot of space. Would be good for Hal and Chris H to see this comment.}
\hal{agree. this section is about 3 pages long. i would suggest moving low-level details to an appendix which can be submitted as supplemental. i've added toappendix.sty to make this easy. i've given an example usage in the RNN encoder section which i think you can use throughout. i also added a makefile that will split the paper into main and supllemental}

For a typical deep learning model of extractive 
summarization there are two main design decisions:
%At a high level, all the models considered in this paper share the same two part structure: 
\textit{i)}  the choice of \textit{sentence encoder} 
which maps each sentence \sent[i] 
%(treated as a sequence of word embeddings) 
to an embedding $\sentEmb[i] \in \mathcal{R}^{\sentEmbSize}$, 
%\hal{notation class, you used $d$ already for number of sentences} 
and 
\textit{ii)} the choice of \textit{sentence extractor} 
which maps a sequence of sentence embeddings 
$\sentEmb = \sentEmb[1],\ldots, \sentEmb[\docSize]$  
to a sequence of extraction
decisions $\slabel = \slabel_1,\ldots,\slabel_{\docSize}$.
The sentence extractor is then a discriminative 
classifier $p(\slabel | \sentEmb)$.
%and predicts which sentences to extract to produce the 
%extract summary. 

We study three architectures for the sentence encoders, namely, 
embedding averaging, RNNs, and 
CNNs.
We also propose two simple models for the sentence extractor and compare
to the previously proposed extractors of 
\citet{cheng2016neural} and \citet{nallapati2017summarunner}.
\hal{i think it's still confusing what's new and what's not. maybe you can somewhat mark? like things with $\star$ are new and ones without are old or something?}
The prior works differ significantly but make the same semi-Markovian
factorization of the extraction decisions, i.e. 
$p(\slabel|\sentEmb)=\prod_{i=1}^\docSize p(\slabel[i]|\slabel[<i],\sentEmb)$,
where each prediction \slabel[i] is dependent on all previous \slabel[j] for
all $j < i$.
By contrast, our extractors make a stronger conditional independence 
assumption $p(\slabel|\sentEmb)=\prod_{i=1}^\docSize p(\slabel[i]|\sentEmb)$,
essentially making independent predictions conditioned on $\sentEmb$.
In theory, our models should perform worse because of this, however, as
we later show, this is not the case empirically.



\hal{i think you might need a subsection at the end of this section with oen or two paragraphs of compare/contrast the different models, esp if details are going to appendix}


%Depending on the architectural choices of each component we propose we 
%can recover the specific implementations of \cite{cheng&lapata} and 
%\cite{nallapati}, which we outline below.

\subsection{Sentence Encoders}
We experiment with three architectures for mapping sequences
of word embeddings to a fixed length vector: averaging, RNNs, and CNNs.
Hyperparameter settings and implementation details can be found 
in \autoref{app:sentencoders}.

\paragraph{Averaging Encoder} Under the averaging encoder, a sentence 
embedding \sentEmb is simply the average of its word embeddings, i.e. 
$\sentEmb = \frac{1}{\sentSize} \sum_{i=1}^{\sentSize} \wordEmb[i]$.

\paragraph{RNN Encoder} When using  the \textit{RNN} sentence encoder,
a sentence embedding is the concatenation of the final output states of a 
forward and backward RNN over the sentence's word embeddings. We use a Gated 
Recurrent Unit (GRU) for the RNN cell \cite{chung2014empirical}.

\paragraph{CNN Encoder} The \textit{CNN} sentence encoder uses a series of 
convolutional feature maps to encode each sentence. This encoder is similar
to the convolutional architecture of \citet{kim2014convolutional} used for 
text classification tasks and performs a series of ``one-dimensional'' 
convolutions over word embeddings. The final sentence embedding $\sentEmb$ is 
a concatenation of all the convolutional filter outputs after max pooling over
time.


%%% Local Variables:
%%% mode: latex
%%% TeX-master: "dlextsum.emnlp18"
%%% End:


\subsection{Sentence Extractors}
%\begin{figure*}
%  \center
%  \includegraphics[scale=.7]{figures/rnnextractor.pdf}
%  \includegraphics[scale=.7]{figures/s2s_extractor.pdf}
%  \includegraphics[scale=.7]{figures/clextractor.pdf}
%  \includegraphics[scale=.7]{figures/rnnextractor.pdf}
%  \caption{Sentence extractor architectures: a) RNN, b) Seq2Seq,
%  c) Cheng \& Lapata, and d) SummaRunner. The $\bigoplus$ indicates 
%  attention. Green repesents sentence encoder output, yellow and orange
%  indicates
%  extractor encoder and decoder hidden states respectively, and red indicates
%  learned ``begin decoding'' embeddings. }
%  \label{fig:extractors}
%\end{figure*}

Given a sequence of sentence embeddings $\sentEmb = \sentEmb[1], \ldots, \sentEmb[\docSize]$ produced by a sentence encoder, 
a sentence extractor defines a conditional distribution $p(\slabel|\sentEmb)$
over the corresponding sentence extraction variables.

We first propose two simple extractor models based on bidirectional RNN 
and sequence-to-sequence with attention architectures, 
which we refer to as the RNN and 
Seq2Seq extractors respectively.

Next we define the extractor models of Cheng \& Lapata, and Nallapati et al.,
whose models we refer to as the Cheng \& Lapata and SummaRunner extractors
respectively.
See Figure~\ref{fig:extractors} for a diagram of the 
four sentence extractor architectures.



%$p(\slabel_1,\ldots,\slabel_{\sentSize}|\sentvec_1, \ldots, \sentvec_{\sentSize})$.
%We propose two simple recurrent neural network based sentence extractors
%that make a strong conditional independence assumption over the labels
%$\slabel_i$, namely
%$\explicitLikelihood= \naiveLikelihood$. This stands in contrast to our 
%baseline models which make a weaker assumption, \hal{this is really confusing cuz i don't think you've specified this yet, and it's still hard to understand what's new/yours and what's old/baseline.}
%$\compactLikelihood = \markovLikelihood$, at the expense of greater 
%computational complexity. 

\paragraph{RNN Extractor}
    Our first proposed model is a very simple bidirectional
RNN based tagging model. As in the RNN sentence encoder we use a GRU cell.
The  forward and backward outputs of each sentence are passed through a 
multi-layer perceptron with a logistic sigmoid output (denoted by $\sigma$)
to predict the probability
of extracting each sentence. See details in Appendix~\ref{app:rnnextractor}

\begin{toappendix}
\section{Details on RNN Extractor.} \label{app:rnnextractor}
\begin{align}
    \rExtHidden_0 = \mathbf{0};&\quad   \rExtHidden_i = \rgru(\sentEmb[i], \rExtHidden_{i-1}) \\
    \lExtHidden_{\docSize + 1} = \mathbf{0};&\quad    \lExtHidden_i = \lgru(\sentEmb[i], \lExtHidden_{i+1}) \\
   \logits_i &= \relu\left(U \cdot [\rExtHidden_i; \lExtHidden_i] + u \right)\\
   p(\slabel_i=1|\sentvec) &= \sigma\left(V\cdot \logits_i + v  \right)
\end{align}
where $\rgru$ and $\lgru$ indicate the 
forward and backward GRUs respectively, and each have separate learned 
parameters; $U, V$ and $u, v$ are learned weight and bias parameters.
\end{toappendix}

\paragraph{\sts~Extractor} One shortcoming of the RNN extractor is that long range
information from one end of the document may not easily be able to effect 
extraction probabilities of sentences at the other end. 
Our second proposed model, the \sts~extractor mitigates this problem with an 
attention 
mechanism commonly
used for neural machine translation \cite{bahdanau2014neural} and 
abstractive summarization \cite{see2017get}. 
The sentence embeddings are first
encoded by a bidirectional $\gru$. A separate decoder $\gru$ transforms each 
sentence into a query vector which attends to the encoder output. The
attention weighted encoder output and the decoder $\gru$ output are concatenated
and fed into a multi-layer percepron to compute the extraction probability.
Detail in Appendix~\ref{app:s2sextractor}.
\begin{toappendix}
\section{Details on Seq2Seq Extractor.} \label{app:s2sextractor}
\begin{align}
    \rEncExtHidden_0 = \textbf{0}&;\quad \rEncExtHidden_i = \rgru_{enc}(\sentEmb[i], \rEncExtHidden_{i-1}) \\
    \lEncExtHidden_{\docSize + 1} = \textbf{0}&;\quad  \lEncExtHidden_i = \lgru_{enc}(\sentEmb[i], \lEncExtHidden_{i+1}) \\
    \rDecExtHidden_i &= \rgru_{dec}(\sentEmb[i], \rDecExtHidden_{i-1}) \\
    \lDecExtHidden_i &= \lgru_{dec}(\sentEmb[i], \lDecExtHidden_{i+1}) 
\end{align}
\begin{align}
 \decExtHidden_i = [\rDecExtHidden_i; \lDecExtHidden_i], &\;\;
 \encExtHidden_i = [\rEncExtHidden_i; \lEncExtHidden_i] 
\end{align}
\begin{align}
 \alpha_{i,j} = 
   \frac{\exp \left(\decExtHidden_i \cdot \encExtHidden_j \right)}{
   \sum_{j=1}^{\docSize}\exp\left(\decExtHidden_i \cdot \encExtHidden_j\right)}, 
& \;\; \attnExtHidden_i = \sum_{j=1}^{\docSize} \alpha_{i,j} \encExtHidden_j 
\end{align}
\begin{align}
   \logits_i = \relu\left(U \cdot [\attnExtHidden_i; \decExtHidden_i] + u \right)&\\
   p(\slabel_i=1|\sentvec) = \sigma\left(V\cdot \logits_i + v  \right).
\end{align}
The final outputs of each encoder direction are passed to the first decoder
steps; additionally, the first step of the decoder GRUs are learned 
``begin decoding'' vectors $\rDecExtHidden_0$ and $\lDecExtHidden_0$ 
(see Figure~\ref{fig:extractors}.b).
Each GRU has separate learned 
parameters; $U, V$ and $u, v$ are learned weight and bias parameters.
\end{toappendix}

\textbf{Cheng \& Lapata Extractor} 
We compare the previously proposed architectures to the sentence extractor
model of \cite{cheng2016neural}. 
%Unlike the previous models where
%sentence extraction predictions are conditionally independent given
%the sentence embeddings, this model uses previous extraction probabilities to
%influence later decisions. 

\begin{toappendix}
\section{Details on Cheng \& Lapata Extractor.} \label{app:clextractor}
The basic architecture is a unidirectional
sequence-to-sequence
model defined as follows:
\begin{align}
    \encExtHidden_0 = \textbf{0};&\quad   \encExtHidden_i = \gru_{enc}(\sentvec_i, \encExtHidden_{i-1}) \\
    \decExtHidden_1 &= \gru_{dec}(\sentEmb[*], \encExtHidden_{\docSize}) \\
    \decExtHidden_i &= \gru_{dec}(p_{i-1} \cdot \sentvec_{i-1}, \decExtHidden_{i-1}) \label{eq:cl1} \\
   \logits_i &= \relu\left(U \cdot [\encExtHidden_i; \decExtHidden_i] + u \right)\\
    p_i = p(\slabel_i&=1|\slabel_{<i}, \sentvec) = \sigma\left(V\cdot \logits_i + v  \right) 
\end{align}
where \sentEmb[*] is a learned ``begin decoding'' sentence embedding
(see Figure~\ref{fig:extractors}.c).
Each GRU has separate learned 
parameters; $U, V$ and $u, v$ are learned weight and bias parameters.
Note in Equation~\ref{eq:cl1} that 
the decoder side GRU input is the sentence embedding from the previous time
step weighted by its probabilitiy of extraction ($p_{i-1}$) from the 
previous step, inducing dependence of each output $y_i$ on all previous 
outputs $y_{<i}$.
\end{toappendix}

%\kathy{In many ways the Cheng and Lapata architecture looks simpler than yours. Can you indicate here why it is more complicated? Is it because of the number of learned parameters?}

Note that in the original paper, the Cheng \& Lapata extractor was paired 
with
a \textit{CNN} sentence encoder, but in this work we experiment with a variety
of sentence encoders.


\paragraph{SummaRunner Extractor}{
Our second baseline, which we refer to as the SummaRunner extractor is taken 
from 
\cite{nallapati2017summarunner}.



\begin{toappendix}
\section{Details on SummaRunner Extractor.} \label{app:srextractor}
Like the
\modelOne~extractor it starts with a bidrectional GRU over the sentence 
embeddings 
\begin{align}
    \rEncExtHidden_0 = \textbf{0}&;\quad \rEncExtHidden_i = \rgru(\sentvec_i, \rEncExtHidden_{i-1}) \\
    \lEncExtHidden_{\docSize + 1} = \textbf{0}&;\quad \lEncExtHidden_i = \lgru(\sentvec_i, \lEncExtHidden_{i+1}),
\end{align}
}

It then creates a representation
of the whole document $q$ by passing the averaged GRU output states through
a fully connected layer: 
\begin{align}
q = \tanh\left(b_q + W_q\frac{1}{\docSize}\sum_{i=1}^{\docSize} [\rEncExtHidden_i; \lEncExtHidden_i] \right)
\end{align}
A concatentation of the GRU outputs at each step
are passed through a separate fully connected layer to create a 
sentence representation $z_i$, where
\begin{align}
    \extHidden_i &= \relu\left(b_z + W_z [\rEncExtHidden_i; \lEncExtHidden_i]\right).
\end{align}
The extraction probability is then determined by contributions from five 
sources:
\begin{align}
    \textit{content} &\quad a^{(con)}_i=W^{(con)} z_i, \\
    \textit{salience}&\quad a^{(sal)}_i = z_i^TW^{(sal)} q, \\
    \textit{novelty}&\quad a^{(nov)}_i = -z_i^TW^{(nov)} \tanh(g_i), \label{eq:srnov} \\
    \textit{position}&\quad a^{(pos)}_i = W^{(pos)} l_i, \\
    \textit{quartile}&\quad a^{(qrt)}_i = W^{(qrt)} r_i,
\end{align}
where $l_i$ and $r_i$ are embeddings associated with the $i$-th sentence
position and the quarter of the document containing sentence $i$ respectively.
In Equation~\ref{eq:srnov}, $g_i$ is an iterative summary representation 
computed as the
sum of the previous $z_{<i}$ weighted by their extraction probabilities,
\begin{align}
g_i & = \sum_{j=1}^{i-1} p(y_j=1|y_{<j},h) \cdot z_j.
\end{align}
Note that the presence of this term induces dependence of each 
$\slabel_i$ to 
all $\slabel_{<i}$ similarly to the Cheng \& Lapata extractor.

The final extraction probability is the logistic sigmoid of the
sum of these terms plus a bias,
\begin{align}
    p(y_i=1|y_{<i}, h) &= \sigma\left(\begin{array}{l}
      a_i^{(con)} + a_i^{(sal)} + a_i^{(nov)} \\
  + a_i^{(pos)}  + a_i^{(qrt)} + b \end{array}\right).
\end{align}
The weight matrices $W_q$, $W_z$, $W^{(con)}$, $W^{(sal)}$, $W^{(nov)}$, $W^{(pos)}$,
$W^{(qrt)}$ and bias terms $b_q$, $b_z$, and $b$ are learned parameters;
The GRUs have separate learned parameters.
\end{toappendix}

Note that in the original paper, the SummaRunner extractor was paired 
with
an \textit{RNN} sentence encoder, but in this work we experiment with a variety
of sentence encoders.

%\begin{align}
%    \extHidden_i &= \relu\left(b_z + W_z [\rEncExtHidden_i; \lEncExtHidden_i]\right)
%  \end{align}
%  \begin{align*}
%      p(y_i=1|y_{<i}, h) = \sigma(& W_{con}\cdot z_i \\
%                     & + z_i^T W_{sal}\cdot q \\
%                     & -z_i^T W_{nov} \cdot \tanh(g_i) \\
%                     & + b_{rp_i}  \\
%                     & + b_{ap_i} \\
%                     & + b)     \\
%      g_j & = \sum_{i=1}^{j-1} p(y_j=1|y_{<j},h) \cdot z_j
%\end{align*}



%%% Local Variables:
%%% mode: latex
%%% TeX-master: "dlextsum.emnlp18"
%%% End:



%%% Local Variables:
%%% mode: latex
%%% TeX-master: "dlextsum.emnlp18"
%%% End:
