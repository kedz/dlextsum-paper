

The results of our main experiment comparing 
the different extractors/encoders are shown in 
Table~\ref{tab:results}.
Overall, we find no major advantage when using the CNN and RNN sentence
encoders. The best performing encoder/extractor pair uses the averaging 
encoder (\textcolor{red}{4 out of 5 datasets}) or the differences 
are not statistically significant. When only comparing within the 
same extractor choice,  the averaging encoder is the better choice
in 14 of 20 cases. 

When looking at extractors, the Seq2Seq extractor is either part of 
the best performing system (\textcolor{red}{2 out of 5} datasets) or is not 
statistically distinguishable from the best extractor. 

Overall, on the news domain, the differences are quite small with the 
differences between worst and best systems on the CNN/DM dataset 
spanning only .56 of a ROUGE point. While there is more performance variability
 in the non-news domains, there is less distinction among systems: all systems
are statistically indistinguishable from the best system on Reddit
and every extractor has at least one configuration that is indistinguishable
from the best system on the AMI corpus.





%?\textcolor{red}{Overall we find that the \modelTwoBF~extractor achieves the 
%?best ROUGE scores on three out of four domains (STILL RUNNING ON AMI AND PUBMED). 
%?However, most
%?differences are not signficant. (Need to discuss stat sig and how to show it).}
%?On the larger CNN-DailyMail dataset, especially, 
%?differences are quite smail across all extractor/encoder pairs.
%?The \baselineOneBF~extractor achieves the best performance on the DUC 2002
%?dataset. It is disappointing that the \baselineOneBF~and \baselineTwoBF~based 
%?models do not gain any apparent advantage in conditioning on previous 
%?sentence selection decisions; this result suggests the need to improve
%?the representation of the summary as it is being constructed iteratively.
%?
%?\textbf{Choice of Encoder} We also find there to be no major advantage 
%?between the different sentence encoders. \textcolor{red}{In most cases,
%?there is no statistical significance between the averaging encoder and either
%?the RNN or CNN encoders.} 

%The lack of differentiation amongst the different encoders concerning; one
%would assume learning with the appropriate structure would be helpful.
%The results of next 




\paragraph{Word Embedding Learning}{

Table~\ref{tab:embeddings} shows ROUGE recall
when using fixed or updated word embeddings. 
\textcolor{red}{In all but one case,
fixed embeddings are as good or better than the learned embeddings.}
This is a somewhat surprising finding, and suggests that our models
cannot extract much generalizable learning signal from the content than what 
is already present from initialization. 
The AMI corpus is an exception here where learning does lead to small
performance boosts, however, only in the Seq2Seq extractor is this diference 
significant. The language of this corpus is quite different from the 
data that the GloVe embeddings were trained on and so it makes sense 
that  there would be more benefit to learning word representations; one
explanation for only seeing modest improvements is purely the small size
of the dataset \textcolor{red}{(NOTE TO CK -- expect learning to help on pubmed)}.


}

\begin{table*}[ht]
\center
\begin{tabular}{cgcgcgc}
    \toprule
    \multirow{1}{*}{\textbf{Ablation}}  & \multicolumn{1}{g}{\textbf{CNN/DM}} & \multicolumn{1}{c}{\textbf{NYT}} & \multicolumn{1}{g}{\textbf{DUC}} & \multicolumn{1}{c}{\textbf{Reddit}} & \multicolumn{1}{g}{\textbf{AMI}} & \multicolumn{1}{c}{\textbf{PubMed}}\\
    \hline
    all words & \textbf{25.4}\textsuperscript{~} ~~~~~~~ & \textbf{34.7}\textsuperscript{~} ~~~~~~~~& 22.7\textsuperscript{~} ~~~~~~~~& \textbf{11.4}\textsuperscript{~} ~~~~~~~~& 5.5\textsuperscript{~} ~~~~~~~~~& \textbf{17.0}\textsuperscript{~} ~~~~~~~ \\
    -nouns & 25.3\textsuperscript{$\dagger$} \footnotesize{(0.1)}& 34.3\textsuperscript{$\dagger$} \footnotesize{(0.4)}& 22.3\textsuperscript{$\dagger$} ~\footnotesize{(0.4)}& 10.3\textsuperscript{$\dagger$} \footnotesize{(1.1)} & 3.8\textsuperscript{$\dagger$} \footnotesize{(1.7)}& 15.7\textsuperscript{$\dagger$} \footnotesize{(1.3)}\\
    -verbs & 25.3\textsuperscript{$\dagger$} \footnotesize{(0.1)}& 34.4\textsuperscript{$\dagger$} \footnotesize{(0.3)} & 22.4\textsuperscript{$\dagger$} ~\footnotesize{(0.3)}& 10.8\textsuperscript{~} ~\footnotesize{(0.6)} & 5.8\textsuperscript{~} \footnotesize{(-0.3)} & 16.6\textsuperscript{$\dagger$} \footnotesize{(0.4)}\\
    -adj/adv & 25.3\textsuperscript{$\dagger$} \footnotesize{(0.1)}& 34.4\textsuperscript{$\dagger$} \footnotesize{(0.3)} & 22.5\textsuperscript{~} ~\footnotesize{(0.2)} & ~~9.5\textsuperscript{$\dagger$} \footnotesize{(1.9)} & 5.4\textsuperscript{~} ~\footnotesize{(0.1)} & 16.8\textsuperscript{$\dagger$} \footnotesize{(0.2)}\\
    -function & 25.2\textsuperscript{$\dagger$} \footnotesize{(0.2)} & 34.5\textsuperscript{$\dagger$} \footnotesize{(0.2)} & \textbf{22.9}\textsuperscript{$\dagger$} \footnotesize{(-0.2)} & 10.3\textsuperscript{$\dagger$} \footnotesize{(1.1)}& \textbf{6.3}\textsuperscript{$\dagger$} \footnotesize{(-0.8)}& 16.6\textsuperscript{$\dagger$} \footnotesize{(0.4)}\\
    \bottomrule
\end{tabular}

\caption{ROUGE-2 recall after removing nouns, verbs, adjectives/adverbs, and 
    function words. Ablations are
    performed using the averaging sentence encoder and the RNN
extractor. 
Bold indicates best performing system. $\dagger$ indicates significant 
difference with the non-ablated system. Difference in score from \textit{all words} shown in parenthesis.}
\label{tab:ablations}
\end{table*}


\textbf{POS Ablation} Table~\ref{tab:ablations} shows the results of the POS
tag ablation experiments. 
While removing any word class from the representation generally hurts 
performance (with statistical significance) on the news domains,
the absolute values of the differences are quite small 
(.18 on CNN/DM, .41 on NYT, .3 on DUC) suggesting that the model's predictions
are not overly dependent on any particular word types. 
On the non-news datasets, the ablations have a larger effect 
(max differences are 1.89 on Reddit, 2.56 on AMI).
Removing the content words leads to the largest drop on AMI.
Removing adjectives and adverbs leads to the largest drop on Reddit,
suggesting the intensifiers and descriptive words are useful for 
identifying important content in personal narratives.
Curiously, 
removing the miscellaneous POS class yields a significant improvement
on DUC 2002 and AMI.


%The newswire domain does not appear to be sensative
%to these ablations; this suggests that the models are still able to identify
%the lead section of the document with the remaining word classes \textcolor{red}{(Verify this with histogram analysis)}. 
%The Reddit domain, which is not lead biased, is significantly effected.
%Notably, removing adjectives and adverbs results in a 1.8 point drop 
%in ROUGE-2 recall. 

\begin{table*}[h]
\center
\begin{tabular}{llgcgcg}
    \toprule
    \multirow{2}{*}{\textbf{Extractor}} &\multirow{2}{*}{\textbf{Sentence Order}}  & \multicolumn{1}{g}{\textbf{CNN/DM}} & \multicolumn{1}{c}{\textbf{NYT}} & \multicolumn{1}{g}{\textbf{DUC 2002}} & \multicolumn{1}{c}{\textbf{Reddit}} & \multicolumn{1}{g}{\textbf{AMI}}\\
     &  & R-2 & R-2 & R-2 & R-2 & R-2\\
    \hline
    \multirow{2}{*}{RNN} & In-Order & \textbf{25.42} & \textbf{34.67} & \textbf{22.65} & \textbf{11.37} & \textbf{5.50}\\
     & Shuffled & 22.80 & 24.96 & 18.24 & \textbf{11.83} & \textbf{5.70}\\
    \hline
    \multirow{2}{*}{Seq2Seq} & In-Order & \textbf{25.56} & \textbf{35.73} & \textbf{22.84} & \textbf{13.61} & 5.52\\
     & Shuffled & 21.66 & 25.61 & 21.21 & \textbf{13.45} & \textbf{5.98}\\
    \bottomrule
\end{tabular}

\caption{ROUGE-2 recall using models trained on in-order and shuffled
documents. All extractors use the averaging sentence encoder. 
Table shows average results of five random initializations.
When both in-order and shuffled settings are bolded,
there is no signifcant performance difference.
}
\label{tab:shuffle}
\end{table*}


%\begin{table*}
%\center
%\begin{tabular}{| c | c || c | c | c | c | c | c |}
%\hline
%  &   & \multicolumn{2}{|c|}{nyt} & \multicolumn{2}{|c|}{duc-sds} & \multicolumn{2}{|c|}{reddit} \\
%Extractor & sentence order & R1 & R2  & R1 & R2  & R1 & R2 \\ 
%\hline
%\multirow{2}{*}{RNN} & in-order & 51.4 & 34.7 & 44.1 & 22.6 & 45.2 & 11.4\\ \cline{2-8}
% & shuffled & 41.9 & 25.0 & 39.7 & 18.2 & 45.1 & 11.9\\
%\hline
%\multirow{2}{*}{Seq2Seq} & in-order & 52.5 & 35.7 & 44.4 & 22.8 & 49.1 & 13.6\\ \cline{2-8}
% & shuffled & 42.6 & 25.6 & 42.9 & 21.2 & 48.7 & 13.6\\
%\hline
%\end{tabular}
%\caption{ROUGE 1 and 2 recall using models trained on in-order and shuffled
%documents. All extractors use the averaging sentence encoder. 
%Table shows average results of five random initializations.}
%\label{tab:shuffle}
%\end{table*}

\textbf{Document Shuffling} Table~\ref{tab:shuffle} shows the results
of our shuffling experiments. 
The newswire domain suffer a significant drop in performance 
when the document order is shuffled. \textcolor{red}{By comparison, there is no significant difference between the shuffled and in-order models on 
the Reddit domain, and shuffling actually improves performance on the AMI}.





\begin{table*}
\begin{tabular}{lllrrrrr}
%\toprule
    &     &          & cnn-dailymail &      nyt &  duc-sds &   reddit &     ami \\
    &     &          &            R2 &       R2 &       R2 &       R2 &      R2 \\
Source Dataset & Extractor & Doc Order &               &          &          &          &         \\
%\midrule
cnn-dailymail & RNN & In-Order &       25.4172 &  32.6914 &  22.7538 &  14.5222 &  4.0718 \\
nyt & RNN & In-Order &       24.1500 &  34.6666 &  22.9564 &  11.2408 &  4.2336 \\
duc-sds & RNN & In-Order &       23.5106 &  32.2476 &  22.6520 &  12.3368 &  3.0092 \\
reddit & RNN & In-Order &       23.0648 &  27.9204 &  20.6258 &  11.3940 &  2.5646 \\
ami & RNN & In-Order &        9.6806 &   7.7956 &  12.6174 &  10.7048 &  5.4958 \\
cnn-dailymail & RNN & Shuffled &       22.8038 &  25.5584 &  21.3410 &  \textbf{14.7008} &  3.4252 \\
nyt & RNN & Shuffled &       17.5148 &  24.9578 &  20.6572 &  10.4836 &  3.8924 \\
duc-sds & RNN & Shuffled &       16.9730 &  18.6806 &  18.2378 &  11.7118 &  3.2894 \\
reddit & RNN & Shuffled &       22.7654 &  27.7366 &  20.5632 &  11.8348 &  2.6030 \\
ami & RNN & Shuffled &        9.9302 &   8.2276 &  12.7670 &  10.2996 &  5.7002 \\
cnn-dailymail & Seq2Seq & In-Order &       \textbf{25.5560} &  32.5312 &  22.7042 &  14.3916 &  3.8886 \\
nyt & Seq2Seq & In-Order &       24.4888 &  \textbf{35.7280} &  \textbf{23.0692} &  10.7674 &  3.5056 \\
duc-sds & Seq2Seq & In-Order &       24.3412 &  32.7244 &  22.8384 &  12.9758 &  3.5844 \\
reddit & Seq2Seq & In-Order &       18.7552 &  23.7096 &  19.8706 &  13.7194 &  3.0072 \\
ami & Seq2Seq & In-Order &       12.1790 &   9.7512 &  13.7606 &   9.9028 &  5.5226 \\
cnn-dailymail & Seq2Seq & Shuffled &       21.6618 &  22.6544 &  19.9770 &  14.0084 &  3.8104 \\
nyt & Seq2Seq & Shuffled &       18.2358 &  25.6094 &  20.5528 &  12.0216 &  3.6208 \\
duc-sds & Seq2Seq & Shuffled &       19.6998 &  25.5266 &  21.2116 &  12.0884 &  3.4744 \\
reddit & Seq2Seq & Shuffled &       19.4136 &  23.8588 &  19.8218 &  13.4550 &  2.8182 \\
ami & Seq2Seq & Shuffled &       13.5160 &  13.3164 &  15.6102 &  10.2888 &  \textbf{5.9806} \\
%\bottomrule
\end{tabular}



\end{table*}

