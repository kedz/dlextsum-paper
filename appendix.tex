\begin{centering}
  \Large
  Supplementary Material For:\\
  Content Selection in Deep Learning Models of Summarization\\

\end{centering}

  \section{Details on Sentence Encoders} \label{app:sentencoders}

  We use 200 dimenional word embeddings \wordEmb[i] in all models.
  Dropout is applied to the embeddings during training.

\subsection{Details on RNN Encoder} \label{app:rnnencoder}
  
Under the \textit{RNN} encoder, a sentence embedding is defined as
$\sentEmb = [\rSentEmb[\sentSize]; \lSentEmb[1]]$
where 
\begin{align} 
  \rSentEmb[0] = \mathbf{0} &;\quad 
       \rSentEmb[i] = \rgru(\wordEmb[i], \rSentEmb[i-1]) \\
  \lSentEmb[\sentSize + 1] = \mathbf{0} &;\quad 
       \lSentEmb[i] = \lgru(\wordEmb[i], \lSentEmb[i+1]),
%\lSentVec_i &= \lgru(w_i, \lSentVec_{i+1}) 
\end{align}
and $\rgru$ amd $\lgru$ indicate the 
forward and backward GRUs respectively, each with separate 
parameters. We use 300 dimensional hidden layers for each GRU. 
Dropout is applied to GRU during training.

\subsection{Details on CNN Encoder} \label{app:cnnencoder}
The \textit{CNN} encoder has hyperparameters
associated with the window sizes $\maxWindowSize \subset \mathcal{N}$ of the convolutional filter 
(i.e. the number of words associated with each convolution) and the number of 
feature maps $\maxFeatureMaps$ associated with each filter
(i.e. the output dimension of each 
convolution). 
The \textit{CNN} sentence embedding $\sentEmb$ is computed as follows:
\begin{align}
 \specActivation_i &= \specConvBias 
    + \sum^\filterWindowSize_{j=1} \specConvWeight_j \cdot \wemb_{i + j -1}\\
  \specFeatureMap &= \max_{i\in 1,\dots, |\sent| - \filterWindowSize + 1} 
                      \relu\left(\specActivation_i\right) \\
 \sentEmb &= \left[\specFeatureMap | 
   \numFeatureMaps \in \{1, \ldots, \maxFeatureMaps\},
   \filterWindowSize \in \maxWindowSize
   \right]
\end{align}
where $\specConvBias\in\mathcal{R}$ and $\specConvWeight \in 
\mathcal{R}^{\filterWindowSize \times \wordEmbSize}$ are learned bias and filter
weight parameters respectively, and $\relu(x) = \max(0, x)$ is the rectified
linear unit activation.
We use window sizes $K={1, 2, 3, 4, 5, 6}$ with corresponding feature maps sizes 25, 25, 50, 50, 50, 50, giving $h$ a dimensionality of 300. 
Dropout is applied to the CNN output during training.

\section{Details on Sentence Extractors} \label{app:sentextractors}
\subsection{Details on RNN Extractor} \label{app:rnnextractor}
\begin{align}
    \rExtHidden_0 = \mathbf{0};&\quad   \rExtHidden_i = \rgru(\sentEmb[i], \rExtHidden_{i-1}) \\
    \lExtHidden_{\docSize + 1} = \mathbf{0};&\quad    \lExtHidden_i = \lgru(\sentEmb[i], \lExtHidden_{i+1}) \\
   \logits_i &= \relu\left(U \cdot [\rExtHidden_i; \lExtHidden_i] + u \right)\\
   p(\slabel_i=1|\sentvec) &= \sigma\left(V\cdot \logits_i + v  \right)
\end{align}
where $\rgru$ and $\lgru$ indicate the 
forward and backward GRUs respectively, and each have separate learned 
parameters; $U, V$ and $u, v$ are learned weight and bias parameters.
The hidden layer size of the GRU is 300 for each direction and the MLP hidden layer
size is 100. Dropout is applied to the GRUs and to $a_i$.

\subsection{Details on Seq2Seq Extrator} \label{app:s2sextractor}
\begin{align}
    \rEncExtHidden_0 = \textbf{0}&;\quad \rEncExtHidden_i = \rgru_{enc}(\sentEmb[i], \rEncExtHidden_{i-1}) \\
    \lEncExtHidden_{\docSize + 1} = \textbf{0}&;\quad  \lEncExtHidden_i = \lgru_{enc}(\sentEmb[i], \lEncExtHidden_{i+1}) \\
    \rDecExtHidden_i &= \rgru_{dec}(\sentEmb[i], \rDecExtHidden_{i-1}) \\
    \lDecExtHidden_i &= \lgru_{dec}(\sentEmb[i], \lDecExtHidden_{i+1}) 
\end{align}
\begin{align}
 \decExtHidden_i = [\rDecExtHidden_i; \lDecExtHidden_i], &\;\;
 \encExtHidden_i = [\rEncExtHidden_i; \lEncExtHidden_i] 
\end{align}
\begin{align}
 \alpha_{i,j} = 
   \frac{\exp \left(\decExtHidden_i \cdot \encExtHidden_j \right)}{
   \sum_{j=1}^{\docSize}\exp\left(\decExtHidden_i \cdot \encExtHidden_j\right)}, 
& \;\; \attnExtHidden_i = \sum_{j=1}^{\docSize} \alpha_{i,j} \encExtHidden_j 
\end{align}
\begin{align}
   \logits_i = \relu\left(U \cdot [\attnExtHidden_i; \decExtHidden_i] + u \right)&\\
   p(\slabel_i=1|\sentvec) = \sigma\left(V\cdot \logits_i + v  \right).
\end{align}
The final outputs of each encoder direction are passed to the first decoder
steps; additionally, the first step of the decoder GRUs are learned 
``begin decoding'' vectors $\rDecExtHidden_0$ and $\lDecExtHidden_0$ 
(see Figure~\ref{fig:extractors}.b).
Each GRU has separate learned 
parameters; $U, V$ and $u, v$ are learned weight and bias parameters.
The hidden layer size of the GRU is 300 for each direction and MLP hidden layer
size is 100. Dropout is applied to the GRUs and to $a_i$.

\subsection{Details on Cheng \& Lapata Extractor.} \label{app:clextractor}
The basic architecture is a unidirectional
sequence-to-sequence
model defined as follows:
\begin{align}
    \encExtHidden_0 = \textbf{0};&\quad   \encExtHidden_i = \gru_{enc}(\sentvec_i, \encExtHidden_{i-1}) \\
    \decExtHidden_1 &= \gru_{dec}(\sentEmb[*], \encExtHidden_{\docSize}) \\
    \decExtHidden_i &= \gru_{dec}(p_{i-1} \cdot \sentvec_{i-1}, \decExtHidden_{i-1}) \label{eq:cl1} \\
   \logits_i &= \relu\left(U \cdot [\encExtHidden_i; \decExtHidden_i] + u \right)\\
    p_i = p(\slabel_i&=1|\slabel_{<i}, \sentvec) = \sigma\left(V\cdot \logits_i + v  \right) 
\end{align}
where \sentEmb[*] is a learned ``begin decoding'' sentence embedding
(see Figure~\ref{fig:extractors}.c).
Each GRU has separate learned 
parameters; $U, V$ and $u, v$ are learned weight and bias parameters.
Note in Equation~\ref{eq:cl1} that 
the decoder side GRU input is the sentence embedding from the previous time
step weighted by its probabilitiy of extraction ($p_{i-1}$) from the 
previous step, inducing dependence of each output $y_i$ on all previous 
outputs $y_{<i}$.
The hidden layer size of the GRU is 300 and the MLP hidden layer
size is 100. Dropout is applied to the GRUs and to $a_i$.


%\kathy{In many ways the Cheng and Lapata architecture looks simpler than yours. Can you indicate here why it is more complicated? Is it because of the number of learned parameters?}

Note that in the original paper, the Cheng \& Lapata extractor was paired 
with
a \textit{CNN} sentence encoder, but in this work we experiment with a variety
of sentence encoders.

\subsection{Details on SummaRunner Extractor.} \label{app:srextractor}
Like the
RNN~extractor it starts with a bidrectional GRU over the sentence 
embeddings 
\begin{align}
    \rEncExtHidden_0 = \textbf{0}&;\quad \rEncExtHidden_i = \rgru(\sentvec_i, \rEncExtHidden_{i-1}) \\
    \lEncExtHidden_{\docSize + 1} = \textbf{0}&;\quad \lEncExtHidden_i = \lgru(\sentvec_i, \lEncExtHidden_{i+1}),
\end{align}

It then creates a representation
of the whole document $q$ by passing the averaged GRU output states through
a fully connected layer: 
\begin{align}
q = \tanh\left(b_q + W_q\frac{1}{\docSize}\sum_{i=1}^{\docSize} [\rEncExtHidden_i; \lEncExtHidden_i] \right)
\end{align}
A concatentation of the GRU outputs at each step
are passed through a separate fully connected layer to create a 
sentence representation $z_i$, where
\begin{align}
    \extHidden_i &= \relu\left(b_z + W_z [\rEncExtHidden_i; \lEncExtHidden_i]\right).
\end{align}
The extraction probability is then determined by contributions from five 
sources:
\begin{align}
    \textit{content} &\quad a^{(con)}_i=W^{(con)} z_i, \\
    \textit{salience}&\quad a^{(sal)}_i = z_i^TW^{(sal)} q, \\
    \textit{novelty}&\quad a^{(nov)}_i = -z_i^TW^{(nov)} \tanh(g_i), \label{eq:srnov} \\
    \textit{position}&\quad a^{(pos)}_i = W^{(pos)} l_i, \\
    \textit{quartile}&\quad a^{(qrt)}_i = W^{(qrt)} r_i,
\end{align}
where $l_i$ and $r_i$ are embeddings associated with the $i$-th sentence
position and the quarter of the document containing sentence $i$ respectively.
In Equation~\ref{eq:srnov}, $g_i$ is an iterative summary representation 
computed as the
sum of the previous $z_{<i}$ weighted by their extraction probabilities,
\begin{align}
g_i & = \sum_{j=1}^{i-1} p(y_j=1|y_{<j},h) \cdot z_j.
\end{align}
Note that the presence of this term induces dependence of each 
$\slabel_i$ to 
all $\slabel_{<i}$ similarly to the Cheng \& Lapata extractor.

The final extraction probability is the logistic sigmoid of the
sum of these terms plus a bias,
\begin{align}
    p(y_i=1|y_{<i}, h) &= \sigma\left(\begin{array}{l}
      a_i^{(con)} + a_i^{(sal)} + a_i^{(nov)} \\
  + a_i^{(pos)}  + a_i^{(qrt)} + b \end{array}\right).
\end{align}
The weight matrices $W_q$, $W_z$, $W^{(con)}$, $W^{(sal)}$, $W^{(nov)}$, $W^{(pos)}$,
$W^{(qrt)}$ and bias terms $b_q$, $b_z$, and $b$ are learned parameters;
The GRUs have separate learned parameters.
The hidden layer size of the GRU is 300 for each direction $z_i$, $q$, and $g_i$ have 100 dimensions. The position and quartile embeddings are 16 dimensional each.
Dropout is applied to the GRUs and to the MLP layer.




Note that in the original paper, the SummaRunner extractor was paired 
with
an \textit{RNN} sentence encoder, but in this work we experiment with a variety
of sentence encoders.


\section{Ground Truth Extract Summary Algorithm} \label{app:oracle}
\begin{algorithmic}

    \State \textbf{data:} $I = \{\sent[1], \ldots, \sent[n]\}$;
    \State $y_i = 0 \quad \forall i \in 1, \ldots, n$;
\State    empty summary $\summary = \emptyset$; ~~word budget $c$
\While{$\sum_{\sent[i] \in \summary} |\sent[i]| \le c \;$} 
\State $\hat{i} = {\argmax}_{ i \in \{j \in [n]| \sent[j] \notin \summary\}} \rouge(\summary \cup \{\sent[i]\})$
\If {$\rouge(\summary \cup \{ \sent[\hat{i}] \} ) > \rouge(\summary)$}
\State $\summary \gets \summary \cup \{ \sent[\hat{i}] \}; \quad y_{\hat{i}} = 1$
        \Else
        \State \textbf{break} 
       \EndIf
\EndWhile
\State \Return $y=y_1, \ldots, y_n$
\end{algorithmic}
\hal{is this right? this could yield something longer than the budget size because what if the last step selects a really long sentence.
  i think you want the argmax over $\{j \in [n] - S : \sum \dots \leq c\}$ to make things brief maybe define $\textit{len}$ to be the current summary length? or $\textit{len}(S)$? also minor quibble but i like $S \cup \{s_i\}$ better than without the braces. CK this is what I did. I saved all summary length constraint for the eval, e.g. rouge just chops things off at 100 words. Not ideal but in practice i think it helps to have more to predict, ie less positive label sparseness, since we are no where near accurately predicting these labels anyway.}



\section{Optimizer and initialization settings.} \label{app:optset}

    We use a learning rate of .0001 and a dropout rate of .25 for all dropout
    layers. We also employ gradient clipping ($-5 < \nabla_\theta < 5$).
Weight matrix parameters are initialized using 
    Xavier initialization with the normal distribution 
    \cite{glorot2010understanding} and bias terms are set to 0.
    We use a batch size of 32 for all datasets except AMI and PubMed, which
    are often longer and consume more memory, for
    which we use sizes two and four respectively.
%\kathy{why? Say.}
    For the Cheng \& Lapata model, we train for half of the maximum epochs 
    with teacher forcing, i.e. we set $p_i = 1$
    if $y_i = 1$ in the gold data and 0 otherwise 
    when computing the decoder input 
    $p_i \cdot \sentEmb[i]$; we revert to the true model probability 
    during the second half training.
\hal{i feel like this paragraph was produced by a summarization algorithm run on the state of Chris' brain :P --- that is to say, it's not super coherant. maybe reorder things so that they are in order of train/apply/eval or something? right now i'm not sure what matters and what doesn't matter. also so-called teacher forcing breaks the definition of log likelihood above, but maybe that's fine.}




