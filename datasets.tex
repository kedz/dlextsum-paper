We perform our experiments across six corpora from varying domains to 
understand how different biases within each domain can affect content 
selection. The corpora come from the news domain
(CNN-DailyMail, New York Times, DUC), personal narratives domain (Reddit),
workplace meetings (AMI), and medical journal articles (PubMed).

\paragraph{CNN-DailyMail} We use the preprocessing and training, validation, 
and test splits
of \cite{see2017get} yielding 287,113/13,368/11,490 documents respectively, each with one reference
abstract. This corpus is a mix of news on different topics including politics,
sports, and celebrity news.

\paragraph{New York Times}{The New York Times (NYT) corpus \cite{sandhaus2008new} contains
\kathy{Not sure why you have this in red? Assume it won't stay in red for the final version?}
 two types of abstracts for a subset of its articles. The first summary is
an abstract \textcolor{red}{produced by an archival librarian} and the 
second is an online teaser meant to elicit a viewer on the webpage to
click to read more. From this collection we take all articles that have 
a combined summary length of at least 100 words. This collection
includes both straight newswire as well as opinion and long-form journalism.
We create training, validation, and test splits by partitioning on dates;
we use the year 2005 as the validation data, with training and test partitions
including documents before and after 2005 respectively,
yielding 44,382/5,523/6,495 documents.}

\paragraph{DUC}{We use the single document summarization data from the 2001
and 2002
Document Understanding Conferences (DUC) \cite{over2002introduction}. We split the 2001 data into training
and validation splits and reserve the 2002 data for testing, resulting in
516/91/657 documents for training, validation, and test respectively. 
The test set has two or three human abstracts roughly 100 words in length per 
articles.}

\paragraph{AMI}{The AMI corpus \cite{carletta2005ami} 
is collection of real and staged office meetings
annotated with text transcriptions, along with abstractive
summaries. We use the proscribed splits to get 98/19/20 training, validation,
and test examples with one human abstract summary per meeting. 
We ignore any speaker information since we are primarily
interested in studying content selection in a domain agnostic way.
The summaries are about 290 words long on average and so we target this length
for summary generation.
}

\paragraph{Reddit}{\citet{ouyang2017crowd} collected a corpus of personal 
    stories shared
 on Reddit\footnote{\url{www.reddit.com}} along with multiple extractive 
 and abstractive summaries. These stories are relatively short compared
 to the other corpora with an average sentence length of ??. 
\kathy{Why didn't you use theirs?}
 We created our own train, validation, and test splits resulting in 
404/24/48 documents respectively. 
}

\paragraph{PubMed}{We created a corpus of 25,000 randomly samples 
    medical journal articles from the PubMed Open Access 
    Subset\footnote{\url{https://www.ncbi.nlm.nih.gov/pmc/tools/openftlist/}}.
    We only included articles if they were at least 1000 words long and 
    had an abstract of at least 50 words in length.
    We used 21250/1250/2500 document for training, validation, test 
    respectively. 
We used the article abstracts as the ground truth human summaries.}



